\documentclass{prbeamer}

% -------------------
% 基本信息
% -------------------
\title{Beamer 模板功能测试}
\author{测试作者}
\institute{某某大学数学系}
\date{\today}

% -------------------
% 参考文献
% -------------------
\addbibresource{ref.bib}

\begin{document}

% 标题页
\begin{frame}
  \titlepage
\end{frame}

% 总目录
\begin{frame}
  \frametitle{总目录}
  \tableofcontents
\end{frame}

%================================================
\section{基础功能测试}
%================================================

\begin{frame}
\frametitle{普通文本与列表}

这是普通正文。测试{\em{这是质量能量方程$E=mc^2$}} 效果。\footnote{这是一个脚注。}

\begin{itemize}
\item 一级条目
\item 二级条目
  \begin{itemize}
  \item 子条目
  \end{itemize}
\end{itemize}

\end{frame}

%------------------------------------------------
\begin{frame}
\frametitle{Block 环境测试}

\begin{block}{普通 Block}
这是一个普通的 block 内容。
\end{block}

\end{frame}

%================================================
\section{定理环境测试}
%================================================

\begin{frame}
\frametitle{定理类环境}

\begin{mytheorem}
设 $a,b\in\mathbb{R}$,则
\[
(a+b)^2=a^2+2ab+b^2.
\]
\end{mytheorem}

\begin{mycorollary}
若 $a=b$,则 $(2a)^2=4a^2$。
\end{mycorollary}

\begin{mylemma}
任意正数平方仍为正数。
\end{mylemma}

\end{frame}

%------------------------------------------------
\begin{frame}
\frametitle{定义与例}

\begin{mydefinition}
若函数 $f$ 在点 $x_0$ 处满足
\[
\lim_{x\to x_0}f(x)=f(x_0),
\]
则称 $f$ 在 $x_0$ 连续。
\end{mydefinition}

\begin{myexample}
函数 $f(x)=x^2$ 在 $\mathbb{R}$ 上处处连续。
\end{myexample}

\end{frame}

%------------------------------------------------
\begin{frame}
\frametitle{命题与证明}

\begin{proposition}
若 $n$ 为偶数,则 $n^2$ 也是偶数。
\end{proposition}

\begin{myproof}
设 $n=2k$,则
\[
n^2=(2k)^2=4k^2=2(2k^2),
\]
故为偶数。
\end{myproof}

\end{frame}

%------------------------------------------------
\begin{frame}
\frametitle{其他定理环境}

\begin{conjecture}
所有偶数都可以表示为两个素数之和。
\end{conjecture}

\begin{remark}
这是一个注记环境。
\end{remark}

\begin{convention}
本文中默认 $\mathbb{N}=\{0,1,2,3,\dots\}$。
\end{convention}

\begin{exercise}
证明:若 $x>0$,则 $\sqrt{x^2}=x$。
\end{exercise}

\end{frame}

%================================================
\section{提示与习题环境}
%================================================

\begin{frame}
\frametitle{hint 与 tips}

这是一个提示测试
\begin{hint}
考虑平方展开公式。
\end{hint}

\begin{tips}
这是一个阅读提示框,可以放较长的说明性文字。
例如:本节内容是后续定理的重要基础。
\end{tips}

\end{frame}

%------------------------------------------------




%------------------------------------------------
\begin{frame}
\frametitle{引用测试}

经典教材可参见 \cite{key111}。
\end{frame}

\section{参考文献}
\begin{frame}
\printbibliography

\end{frame}


\end{document}
